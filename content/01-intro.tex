\chapter{Introduction}
\label{ch:intro}

This document uses several abbreviations, including the \gls{gps}, \gls{api}, and \gls{ram}.

The \gls{gps} is a satellite-based navigation system, the \gls{api} is a set of tools for building software applications, and \gls{ram} is a type of computer memory.


\section{Citations}
\label{sec:citations}
Details about citations are in \cref{sec:citations}.
This is a citation~\cite{wagner2024optimized}. 
Here is a textual citation that shows the author's name: \textcite{wagner2024optimized}. 
Here is an author-only citation: \citeauthor{wagner2024optimized} did something useful.
You can also specify page numbers and extra info \cite[see][p.~5]{wagner2024optimized}. 

This is an example of multiple citations \cite{Bocklet2014ErlangenCLPLarge, Bocklet2013AutomaticPhonemeAnalysis, Bocklet2009TowardsLanguageindependenta}.


\section{URLs}
\href{https://www.example.com}{Click here to open example.com}. You can change the \texttt{urlcolor} configuration in the preamble if you prefer a different URL color.

\section{Tables}
As shown in \cref{tab:example}, the data is easy to understand.
\begin{table}[ht]
    \centering
    \begin{tabular}{lcr}
        \toprule
        \textbf{Column 1} & \textbf{Column 2} & \textbf{Column 3} \\ 
        \midrule
        Apple             & 10                & \$1.00            \\
        Banana            & 5                 & \$0.50            \\
        Cherry            & 20                & \$3.00            \\
        \midrule
        Total             & 35                & \$4.50            \\
        \bottomrule
    \end{tabular}
    \caption{An example table using the \texttt{booktabs} package.}
    \label{tab:example}
\end{table}

\section{Figures}
The \texttt{subcaption} package provides support for adding subfigures or subtables within a figure or table environment. 
This is particularly useful when you want to include multiple images, plots, or tables as part of a larger group, each with its own label and caption. 
This is a reference to \cref{fig:mainfig}.
\begin{figure}[ht]
    \centering
    % Subfigure 1
    \begin{subfigure}{0.45\textwidth}
        \centering
        \includegraphics[width=\linewidth]{example-image-a}
        \caption{First subfigure}
        \label{fig:subfig1}
    \end{subfigure}
    % Subfigure 2
    \begin{subfigure}{0.45\textwidth}
        \centering
        \includegraphics[width=\linewidth]{example-image-b}
        \caption{Second subfigure}
        \label{fig:subfig2}
    \end{subfigure}
    % Main caption
    \caption{Main figure caption describing both subfigures.}
    \label{fig:mainfig}
\end{figure}

\section{Listings}
\Cref{lst:example} is a listing:
\begin{lstlisting}[
    language=Python,
    basicstyle=\ttfamily\footnotesize,
    keywordstyle=\color{blue}\bfseries, 
    commentstyle=\color{green}\itshape, 
    stringstyle=\color{red},            
    numbers=left,
    numberstyle=\tiny,
    stepnumber=1,
    frame=single,
    caption={Example},
    label={lst:example},
    captionpos=b,
    breaklines=true,
    tabsize=2,
    showstringspaces=false
]
import sympy as sp

def approaching_zero(epsilon=1e-10):
    """
    Calculates the limit of 1/x as x approaches zero from the positive side.
    """
    x = sp.Symbol('x', real=True, positive=True)
    limit_expr = 1 / x

    # Calculate the limit
    limit_result = sp.limit(limit_expr, x, epsilon)

    if limit_result == sp.oo:
        return "As x approaches zero, 1/x approaches... infinity! (wow!)"
    elif limit_result == -sp.oo:
        return "Wait, how did we get negative infinity?"
    else:
        return f"The limit is {limit_result}."

print(approaching_zero())
\end{lstlisting}

\section{Algorithms}
This is an algorithm: 

\begin{algorithm}
\caption{Approaching Zero}\label{alg:approaching_zero}
\begin{algorithmic}[1]
\Require Epsilon ($\epsilon$) $\gets 1 \times 10^{-10}$
\State Define a symbol $x$ such that $x > 0$
\State Define the expression $L \gets \frac{1}{x}$
\State Compute the limit as $x \to \epsilon$: $\text{limit\_result} \gets \text{limit}(L, x, \epsilon)$
\If {$\text{limit\_result} = \infty$}
    \State Return "As $x$ approaches zero, $1/x$ approaches infinity!"
\ElsIf {$\text{limit\_result} = -\infty$}
    \State Return "Wait, how did we get negative infinity?"
\Else
    \State Return "The limit is $\text{limit\_result}$. That was unexpected!"
\EndIf
\end{algorithmic}
\end{algorithm}


